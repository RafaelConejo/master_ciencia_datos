\section{Apéndice C. Código Regresión sobre California}

\begin{lstlisting}[language=R]
#Cargo librerias
library(kknn)
library(dplyr)
library(ggplot2)
library(corrplot)

#Cargo el dataset
df_california <- read.csv("california.dat", comment.char="@", header=F)

#Funcion para poner nombre a las columnas
add_colnames <- function(df) {
	colnames(df) <- c("Longitude", "Latitude", "HousingMedianAge",
	"TotalRooms", "TotalBedrooms", "Population", "Households",
	"MedianIncome", "MedianHouseValue")
	df
}

df_california <- add_colnames(df_california)


# Modelos lineales simples
lm_Longitude <- lm(MedianHouseValue ~ Longitude, data = df_california)
lm_Latitude <- lm(MedianHouseValue ~ Latitude, data = df_california)
lm_HousingMedianAge <- lm(MedianHouseValue ~ HousingMedianAge, data = df_california)
lm_TotalRooms <- lm(MedianHouseValue ~ TotalRooms, data = df_california)
lm_TotalBedrooms <- lm(MedianHouseValue ~ TotalBedrooms, data = df_california)
lm_Population <- lm(MedianHouseValue ~ Population, data = df_california)
lm_Households <- lm(MedianHouseValue ~ Households, data = df_california)
lm_MedianIncome <- lm(MedianHouseValue ~ MedianIncome, data = df_california)





summary(lm_Longitude)
summary(lm_Latitude)
summary(lm_HousingMedianAge)
summary(lm_TotalRooms)
summary(lm_TotalBedrooms)
summary(lm_Population)
summary(lm_Households)
summary(lm_MedianIncome)


#Represento graficamente los modelos
plot(MedianHouseValue ~ Longitude, data = df_california)
abline(lm_Longitude,col="red")

plot(MedianHouseValue ~ Latitude, data = df_california)
abline(lm_Latitude,col="red")

plot(MedianHouseValue ~ HousingMedianAge, data = df_california)
abline(lm_HousingMedianAge,col="red")

plot(MedianHouseValue ~ TotalRooms, data = df_california)
abline(lm_TotalRooms,col="red")

plot(MedianHouseValue ~ TotalBedrooms, data = df_california)
abline(lm_TotalBedrooms,col="red")

plot(MedianHouseValue ~ Population, data = df_california)
abline(lm_Population,col="red")

plot(MedianHouseValue ~ Households, data = df_california)
abline(lm_Households,col="red")

plot(MedianHouseValue ~ MedianIncome, data = df_california)
abline(lm_MedianIncome,col="red")



# Funcion para aplicar lm a 5-fold
run_lm_fold <- function(i, x, model, tt = "test") {
	# Cargar conjuntos de train
	file <- paste(x, "-5-", i, "tra.dat", sep="")
	x_tra <- read.csv(file, comment.char="@")
	
	# Cargar conjuntos de test
	file <- paste(x, "-5-", i, "tst.dat", sep="")
	x_tst <- read.csv(file, comment.char="@")
	
	
	
	x_tra <- add_colnames(x_tra)
	x_tst <- add_colnames(x_tst)
	
	
	if (tt == "train") {
		test <- x_tra
	} else {
		test <- x_tst
	}
	
	# Entrenar el modelo sobre el conjunto de train
	formula <- terms(model)
	model_eval <- lm(formula=formula, data=x_tra)
	
	# RMSE sobre test
	yprime <- predict(model_eval, test)
	#MSE
	sum(abs(test$MedianHouseValue - yprime)^2)/length(yprime) 
}



#Procedo a evaluar todos los modelos lineales
cat('RMSE de 5-fold sobre MedianIncome:', mean(sapply(1:5, run_lm_fold, 'california', lm_MedianIncome), fill=T))

cat('RMSE de 5-fold sobre Latitude:', mean(sapply(1:5, run_lm_fold, 'california', lm_Latitude), fill=T))

cat('RMSE de 5-fold sobre TotalRooms:', mean(sapply(1:5, run_lm_fold, 'california', lm_TotalRooms), fill=T))

cat('RMSE de 5-fold sobre HousingMedianAge:', mean(sapply(1:5, run_lm_fold, 'california', lm_HousingMedianAge), fill=T))

cat('RMSE de 5-fold sobre Households:', mean(sapply(1:5, run_lm_fold, 'california', lm_Households), fill=T))



# Modelo lineal multiple con todas las variables
fit_mult1 <- lm(MedianHouseValue ~ ., data=df_california)
summary(fit_mult1)

cat('Media RMSE del modelo lineal multiple sobre 5-fold:', mean(sapply(1:5, run_lm_fold, 'california', fit_mult1), fill=T))


#Modelo lineal multiple sin Longitude
fit_mult2 <- lm(MedianHouseValue ~ . -Longitude , data=df_california)
summary(fit_mult2)

cat('Media RMSE del modelo lineal multiple sobre 5-fold:', mean(sapply(1:5, run_lm_fold, 'california', fit_mult2), fill=T))


#Modelo lineal multiple sin Population
fit_mult3 <- lm(MedianHouseValue ~ . -Population , data=df_california)
summary(fit_mult3)

cat('Media RMSE del modelo lineal multiple sobre 5-fold:', mean(sapply(1:5, run_lm_fold, 'california', fit_mult3), fill=T))


#Modelo lineal multiple sin TotalBedrooms
fit_mult4 <- lm(MedianHouseValue ~ . -TotalBedrooms , data=df_california)
summary(fit_mult4)

cat('Media RMSE del modelo lineal multiple sobre 5-fold:', mean(sapply(1:5, run_lm_fold, 'california', fit_mult4), fill=T))


#Modelo lineal multiple MedianIncome^2
fit_mult5 <- lm(MedianHouseValue ~ . +I(MedianIncome^2) , data=df_california)
summary(fit_mult5)

cat('Media RMSE del modelo lineal multiple sobre 5-fold:', mean(sapply(1:5, run_lm_fold, 'california', fit_mult5), fill=T))

#Modelo lineal multiple MedianIncome^4
fit_mult6 <- lm(MedianHouseValue ~ . +I(MedianIncome^4) , data=df_california)
summary(fit_mult6)

cat('Media RMSE del modelo lineal multiple sobre 5-fold:', mean(sapply(1:5, run_lm_fold, 'california', fit_mult6), fill=T))

#Modelo lineal multiple I(MedianIncome^4)*Latitude*TotalRooms
fit_mult7 <- lm(MedianHouseValue ~ . +I(MedianIncome^4)*Latitude*TotalRooms , data=df_california)
summary(fit_mult7)

cat('Media RMSE del modelo lineal multiple sobre 5-fold:', mean(sapply(1:5, run_lm_fold, 'california', fit_mult7), fill=T))


#Modelo lineal multiple I(MedianIncome^4)*TotalRooms
fit_mult8 <- lm(MedianHouseValue ~ . +I(MedianIncome^4)*TotalRooms , data=df_california)
summary(fit_mult8)

cat('Media RMSE del modelo lineal multiple sobre 5-fold:', mean(sapply(1:5, run_lm_fold, 'california', fit_mult8), fill=T))


#Modelo lineal multiple I(MedianIncome^4)*TotalRooms*HousingMedianAge*Households
fit_mult9 <- lm(MedianHouseValue ~ . +I(MedianIncome^4)*TotalRooms*HousingMedianAge*Households , data=df_california)
summary(fit_mult9)

cat('Media RMSE del modelo lineal multiple sobre 5-fold:', mean(sapply(1:5, run_lm_fold, 'california', fit_mult9), fill=T))


#Modelo lineal multiple MedianIncome*TotalRooms*HousingMedianAge*Households 
fit_mult10 <- lm(MedianHouseValue ~ . +MedianIncome*TotalRooms*HousingMedianAge*Households , data=df_california)
summary(fit_mult10)

cat('Media RMSE del modelo lineal multiple sobre 5-fold:', mean(sapply(1:5, run_lm_fold, 'california', fit_mult10), fill=T))



#Funcion para aplicar knn sobre 5-folds
run_knn_fold <- function(i, x, formula, k,  tt = "test") {
	# Cargar conjuntos de train
	file <- paste(x, "-5-", i, "tra.dat", sep="")
	x_tra <- read.csv(file, comment.char="@")
	
	# Cargar conjuntos de test
	file <- paste(x, "-5-", i, "tst.dat", sep="")
	x_tst <- read.csv(file, comment.char="@")
	
	
	
	x_tra <- add_colnames(x_tra)
	x_tst <- add_colnames(x_tst)
	
	
	if (tt == "train") {
		test <- x_tra
	} else {
		test <- x_tst
	}
	
	# Entrenar el modelo sobre el conjunto de train
	model_eval <- kknn(formula=formula, x_tra, test, k=k)
	
	# RMSE sobre test
	yprime <- model_eval$fitted.values
	#MSE
	sum(abs(test$MedianHouseValue - yprime)^2)/length(yprime) 
}


#Pruebo con diferentes valores de k
cat('RMSE del modelo k-NN sobre 5-fold:', sapply(1:5, run_knn_fold, 'california', MedianHouseValue ~ . +MedianIncome*TotalRooms*HousingMedianAge*Households, k = 5), fill=T)
cat('Meida RMSE del modelo k-NN sobre 5-fold:', mean(sapply(1:5, run_knn_fold, 'california', MedianHouseValue ~ . +MedianIncome*TotalRooms*HousingMedianAge*Households, k = 5), fill=T))


cat('RMSE del modelo k-NN sobre 5-fold:', sapply(1:5, run_knn_fold, 'california', MedianHouseValue ~ . +MedianIncome*TotalRooms*HousingMedianAge*Households, k = 7), fill=T)
cat('Meida RMSE del modelo k-NN sobre 5-fold:', mean(sapply(1:5, run_knn_fold, 'california', MedianHouseValue ~ . +MedianIncome*TotalRooms*HousingMedianAge*Households, k = 7), fill=T))


cat('RMSE del modelo k-NN sobre 5-fold:', sapply(1:5, run_knn_fold, 'california', MedianHouseValue ~ . +MedianIncome*TotalRooms*HousingMedianAge*Households, k = 10), fill=T)
cat('Meida RMSE del modelo k-NN sobre 5-fold:', mean(sapply(1:5, run_knn_fold, 'california', MedianHouseValue ~ . +MedianIncome*TotalRooms*HousingMedianAge*Households, k = 10), fill=T))


cat('RMSE del modelo k-NN sobre 5-fold:', sapply(1:5, run_knn_fold, 'california', MedianHouseValue ~ . +MedianIncome*TotalRooms*HousingMedianAge*Households, k = 20), fill=T)
cat('Meida RMSE del modelo k-NN sobre 5-fold:', mean(sapply(1:5, run_knn_fold, 'california', MedianHouseValue ~ . +MedianIncome*TotalRooms*HousingMedianAge*Households, k = 20), fill=T))


cat('RMSE del modelo k-NN sobre 5-fold:', sapply(1:5, run_knn_fold, 'california', MedianHouseValue ~ . +MedianIncome*TotalRooms*HousingMedianAge*Households, k = 25), fill=T)
cat('Meida RMSE del modelo k-NN sobre 5-fold:', mean(sapply(1:5, run_knn_fold, 'california', MedianHouseValue ~ . +MedianIncome*TotalRooms*HousingMedianAge*Households, k = 25), fill=T))


cat('RMSE del modelo k-NN sobre 5-fold:', sapply(1:5, run_knn_fold, 'california', MedianHouseValue ~ . +MedianIncome*TotalRooms*HousingMedianAge*Households, k = 50), fill=T)
cat('Meida RMSE del modelo k-NN sobre 5-fold:', mean(sapply(1:5, run_knn_fold, 'california', MedianHouseValue ~ . +MedianIncome*TotalRooms*HousingMedianAge*Households, k = 50), fill=T))


#knn sobre modelo con todas las variables
cat('RMSE del modelo k-NN sobre 5-fold:', sapply(1:5, run_knn_fold, 'california', MedianHouseValue ~ ., k = 20), fill=T)
cat('Meida RMSE del modelo k-NN sobre 5-fold:', mean(sapply(1:5, run_knn_fold, 'california', MedianHouseValue ~ ., k = 20), fill=T))




# Comparativa LM, k-NN y M5'
#leemos la tabla con los errores medios de test
resultados <- read.csv("regr_test_alumnos.csv")
tablatst <- cbind(resultados[,2:dim(resultados)[2]])
colnames(tablatst) <- names(resultados)[2:dim(resultados)[2]]
rownames(tablatst) <- resultados[,1]


#Calculo mis valores de LM y KNN para aniadirlos en la tabla
media_lm <-mean(sapply(1:5, run_lm_fold, 'california', fit_mult1), fill=T)
media_knn <- mean(sapply(1:5, run_knn_fold, 'california', MedianHouseValue ~., k = 20), fill=T)

tablatst["california", 1] <- media_lm
tablatst["california", 2] <- media_knn

#Comparar lm con knn con Wilcoxon
# + 0.1 porque wilcox R falla para valores == 0 en la tabla
difs <- (tablatst[,1] - tablatst[,2]) / tablatst[,1]
wilc_1_2 <- cbind(ifelse (difs<0, abs(difs)+0.1, 0+0.1),
ifelse (difs>0, abs(difs)+0.1, 0+0.1))

colnames(wilc_1_2) <- c(colnames(tablatst)[1], colnames(tablatst)[2])
head(wilc_1_2)

LMvsKNNtst <- wilcox.test(wilc_1_2[,1], wilc_1_2[,2], alternative = "two.sided", paired=TRUE)
Rmas <- LMvsKNNtst$statistic
pvalue <- LMvsKNNtst$p.value

LMvsKNNtst <- wilcox.test(wilc_1_2[,2], wilc_1_2[,1], alternative = "two.sided", paired=TRUE)
Rmenos <- LMvsKNNtst$statistic


cat('Test modelo lineal (R+) vs modelo k-NN:(R-)', fill=T)
cat('Valor R+: ',Rmas, fill=T)
cat('Valor R-: ',Rmenos, fill=T)
cat('p-value del test: ',pvalue, fill=T)


test_friedman <- friedman.test(as.matrix(tablatst))
test_friedman

tam <- dim(tablatst)
groups <- rep(1:tam[2], each=tam[1])
pairwise.wilcox.test(as.matrix(tablatst), groups, p.adjust = "holm", paired = TRUE)
\end{lstlisting}