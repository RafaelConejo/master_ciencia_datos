\chapter{Apéndice}
\section{Apéndice A. Código EDA dataset de regresión 'California'}




\definecolor{miverde}{rgb}{0,0.6,0}
\definecolor{migris}{rgb}{0.5,0.5,0.5}
\definecolor{mimalva}{rgb}{0.58,0,0.82}

\lstset{ %
	backgroundcolor=\color{white},   % Indica el color de fondo; necesita que se añada \usepackage{color} o \usepackage{xcolor}
	basicstyle=\footnotesize,        % Fija el tamaño del tipo de letra utilizado para el código
	breakatwhitespace=false,         % Activarlo para que los saltos automáticos solo se apliquen en los espacios en blanco
	breaklines=true,                 % Activa el salto de línea automático
	captionpos=b,                    % Establece la posición de la leyenda del cuadro de código
	commentstyle=\color{miverde},    % Estilo de los comentarios
	deletekeywords={...},            % Si se quiere eliminar palabras clave del lenguaje
	escapeinside={\%*}{*)},          % Si quieres incorporar LaTeX dentro del propio código
	extendedchars=true,              % Permite utilizar caracteres extendidos no-ASCII; solo funciona para codificaciones de 8-bits; para UTF-8 no funciona. En xelatex necesita estar a true para que funcione.
	frame=false,	                   % Añade un marco al código
	keepspaces=true,                 % Mantiene los espacios en el texto. Es útil para mantener la indentación del código(puede necesitar columns=flexible).
	keywordstyle=\color{blue},       % estilo de las palabras clave
	language=R,                 % El lenguaje del código
	otherkeywords={*,...},           % Si se quieren añadir otras palabras clave al lenguaje
	numbers=none,                    % Posición de los números de línea (none, left, right).
	numbersep=5pt,                   % Distancia de los números de línea al código
	numberstyle=none, % Estilo para los números de línea
	rulecolor=\color{white},         % Si no se activa, el color del marco puede cambiar en los saltos de línea entre textos que sea de otro color, por ejemplo, los comentarios, que están en verde en este ejemplo
	showspaces=false,                % Si se activa, muestra los espacios con guiones bajos; sustituye a 'showstringspaces'
	showstringspaces=false,          % subraya solamente los espacios que estén en una cadena de esto
	showtabs=false,                  % muestra las tabulaciones que existan en cadenas de texto con guión bajo
	stepnumber=2,                    % Muestra solamente los números de línea que corresponden a cada salto. En este caso: 1,3,5,...
	stringstyle=\color{mimalva},     % Estilo de las cadenas de texto
	tabsize=2,	                   % Establece el salto de las tabulaciones a 2 espacios
	title=\lstname                   % muestra el nombre de los ficheros incluidos al utilizar \lstinputlisting; también se puede utilizar en el parámetro caption
}

	



\begin{lstlisting}[language=R]
#Cargo librerias necesarias
library(moments)
library(ggplot2)
library(corrplot)
library(ggpubr)
library(tidyverse)

#Cargo el dataset
california <- read.csv('california.dat', comment.char = '@', 
header = FALSE, stringsAsFactors = TRUE)

#Compruebo dimensiones e info de las variables
dim(california)
str(california)

#Pongo nombre a las variables
names(california) <- c("Longitude", "Latitude", "HousingMedianAge",
"TotalRooms", "TotalBedrooms", "Population", "Households",
"MedianIncome", "MedianHouseValue")

#Asignacion automatica, facilita el acceso a los campos
#n <- length(names(california)) - 1
#names(california)[1:n] <- paste ("X", 1:n, sep="")
#names(california)[n+1] <- "Y"

#Missing values
any(is.na(california))

#muestras duplicadas
sum(duplicated(california))

#Medidad importantes
summary(california)
desviacion = apply(california, 2, sd)
format(desviacion, scientific = FALSE)
skev <- apply(california, 2, skewness)
kurt <- apply(california, 2, kurtosis)
format(skev, scientific = FALSE)
format(kurt, scientific = FALSE)

#Representacion grafica de los diagramas de cajas
ggplot(california, aes(y=Longitude)) +
geom_boxplot(outlier.color = "red")
ggplot(california, aes(y=Latitude)) +
geom_boxplot(outlier.color = "red")
ggplot(california, aes(y=HousingMedianAge)) +
geom_boxplot(outlier.color = "red")
ggplot(california, aes(y=TotalRooms)) +
geom_boxplot(outlier.color = "red")
ggplot(california, aes(y=TotalBedrooms)) +
geom_boxplot(outlier.color = "red")
ggplot(california, aes(y=Population)) +
geom_boxplot(outlier.color = "red")
ggplot(california, aes(y=Households)) +
geom_boxplot(outlier.color = "red")
ggplot(california, aes(y=MedianIncome)) +
geom_boxplot(outlier.color = "red")

ggplot(california, aes(y=MedianHouseValue)) +
geom_boxplot(outlier.color = "red")


#Representacion grafica de los histogramas
ggplot(california, aes(x=Longitude)) +
geom_histogram()
ggplot(california, aes(x=Latitude)) +
geom_histogram()
ggplot(california, aes(x=HousingMedianAge)) +
geom_histogram()
ggplot(california, aes(x=TotalRooms)) +
geom_histogram()
ggplot(california, aes(x=TotalBedrooms)) +
geom_histogram()
ggplot(california, aes(x=Population)) +
geom_histogram()
ggplot(california, aes(x=Households)) +
geom_histogram()
ggplot(california, aes(x=MedianIncome)) +
geom_histogram()

ggplot(california, aes(x=MedianHouseValue)) +
geom_histogram()


#Test Shapiro-Wilk para comprobar normalidad en los datos
shapiro.test(sample(california[ , 'Longitude'], 5000))
shapiro.test(sample(california[ , 'Latitude'], 5000))
shapiro.test(sample(california[ , 'HousingMedianAge'], 5000))
shapiro.test(sample(california[ , 'TotalRooms'], 5000))
shapiro.test(sample(california[ , 'TotalBedrooms'], 5000))
shapiro.test(sample(california[ , 'Population'], 5000))
shapiro.test(sample(california[ , 'Households'], 5000))
shapiro.test(sample(california[ , 'MedianIncome'], 5000))


#Elimino los casos de MedianHouseValue == 500000
ggplot(california, aes(x = MedianIncome , y= MedianHouseValue)) +
geom_point()

california_clean <- drop(california[california$MedianHouseValue < 500000, ])
#Para resetear los indices
rownames(california_clean) <- NULL

ggplot(california_clean, aes(x = MedianIncome , y= MedianHouseValue)) +
geom_point()


# Estudio de los Outliers
ggplot(california_clean, aes(x = TotalRooms  , y = MedianHouseValue, color = TotalRooms > (quantile(TotalRooms, 0.75) + IQR(TotalRooms)*1.5) )) +
geom_point()  +
labs(title="Outliers TotalRooms", color="Outliers")

ggplot(california_clean, aes(x = TotalBedrooms  , y = MedianHouseValue, color = TotalBedrooms > (quantile(TotalBedrooms, 0.75) + IQR(TotalBedrooms)*1.5) )) +
geom_point()  +
labs(title="Outliers TotalBedrooms", color="Outliers")

ggplot(california_clean, aes(x = Population  , y = MedianHouseValue, color = Population > (quantile(Population, 0.75) + IQR(Population)*1.5) )) +
geom_point()  +
labs(title="Outliers Population", color="Outliers")

ggplot(california_clean, aes(x = Households  , y = MedianHouseValue, color = Households > (quantile(Households, 0.75) + IQR(Households)*1.5) )) +
geom_point()  +
labs(title="Outliers Households", color="Outliers")

ggplot(california_clean, aes(x = MedianIncome  , y = MedianHouseValue, color = MedianIncome > (quantile(MedianIncome, 0.75) + IQR(MedianIncome)*1.5) )) +
geom_point()  +
labs(title="Outliers MedianIncome", color="Outliers")


#Elimino los outliers
out_TotalRooms <- quantile(california_clean$TotalRooms, 0.75) + IQR(california_clean$TotalRooms)*1.5

out_TotalBedrooms <- quantile(california_clean$TotalBedrooms, 0.75) + IQR(california_clean$TotalBedrooms)*1.5

out_Population <- quantile(california_clean$Population, 0.75) + IQR(california_clean$Population)*1.5

out_Households <- quantile(california_clean$Households, 0.75) + IQR(california_clean$Households)*1.5



california_new <- drop(california_clean[california_clean$TotalRooms < out_TotalRooms ,])
california_new <- drop(california_clean[california_clean$TotalBedrooms < out_TotalBedrooms ,])
california_new <- drop(california_clean[california_clean$Population < out_Population ,])
california_new <- drop(california_clean[california_clean$Households < out_Households ,])
#Para resetear los indices
rownames(california_new) <- NULL

#Se eliminan un 2% solamente, interesa
dim(california_new)


#Matriz de correlaciones
corr_matrix <- cor(california_new, method = "kendall")
corrplot(corr_matrix, method = "color", tl.col = "black")


#Primera hipotesis
ggplot(california_new, aes(x = Longitude, y = Latitude, color = MedianHouseValue, hue = MedianHouseValue))+
geom_point() +
labs(title="Valor medio de la vivienda dependiendo de la localizacion", color="MedianHouseValue") +
scale_color_gradient(low="blue", high="red")

mapa <- png::readPNG("mapa.png")

ggplot(california_new, aes(x = Longitude, y = Latitude, color = MedianHouseValue, hue = MedianHouseValue))+
background_image(mapa)+
geom_point() +
labs(title="Valor medio de la vivienda dependiendo de la localizacion", color="MedianHouseValue") +
scale_color_gradient(low="blue", high="red")


#Segunda hipotesis
ggplot(california_new, aes(x = Longitude, y = Latitude, color = MedianHouseValue, size = Population))+
background_image(mapa)+
geom_point() +
labs(title="Valor medio de la vivienda dependiendo de la localizacion", color="MedianHouseValue") +
scale_color_gradient(low="blue", high="red")

ggplot(california_new, aes(x = Population, y = MedianHouseValue))+
geom_point(color = "blue") 


#Hipotesis 3
ggplot(california_new, aes(x = MedianIncome, y = MedianHouseValue))+
geom_point(color = "blue") 


#Hipotesis 4
ggplot(california_new, aes(x = HousingMedianAge, y = MedianHouseValue))+
geom_point(color = "blue") 

ggplot(california_new, aes(x = Longitude, y = Latitude, color = HousingMedianAge))+
background_image(mapa)+
geom_point() +
labs(title="Edad media de la vivienda dependiendo de la localizacion", color="HousingMedianAge") +
scale_color_gradient(low="green", high="purple")

ggplot(california_new, aes(x = Longitude, y = Latitude, color = MedianHouseValue, hue = MedianHouseValue))+
background_image(mapa)+
geom_point() +
labs(title="Valor medio de la vivienda dependiendo de la localizacion", color="MedianHouseValue") +
scale_color_gradient(low="blue", high="red")
\end{lstlisting}











