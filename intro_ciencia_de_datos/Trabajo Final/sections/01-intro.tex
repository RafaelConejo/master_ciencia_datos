\chapter{Introducción}
Este documento recoge el informe obtenido con la realización del trabajo final de la asignatura \textit{Introducción a la Ciencia de Datos}. Este consiste en la realización de un análisis exploratorio de los datos pertenecientes a dos \textit{dataset} distintos, con la intención de comprender y determinar información importante sobre las variables que forman a cada conjunto de datos. Tras ello, dependiendo del dataset, se generarán diversos modelos de clasificación o regresión.

\vspace{0.5cm}
El primer dataset tratado es el \textbf{California}, sobre el que se plantea un problema de regresión y la búsqueda de desarrollar diversos modelos óptimos de regresión lineal simple y múltiple, modelos de regresión basados en KNN (K Nearest Neighbour) y modelos de regresión no lineales.
\textbf{Bupa} es el segundo dataset, el cual plantea un problema de clasificación que será abordado con modelos de clasificación basados en KNN, modelos basados en LDA y modelos basados en QDA.

